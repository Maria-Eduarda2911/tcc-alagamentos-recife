\chapter{Introdução}

\section{Contextualização do Problema}

Recife está entre as capitais brasileiras mais expostas a alagamentos urbanos. A combinação de fatores naturais — como a baixa altitude média, o relevo plano e a influência direta das marés — com o crescimento acelerado da cidade, marcado pela impermeabilização do solo e pelas ocupações irregulares, cria um cenário propício para enchentes recorrentes. Pesquisas sobre mudanças climáticas indicam que esse quadro tende a se agravar, com chuvas cada vez mais intensas e frequentes no Nordeste \cite{marengo2009mudancas}.  

Os impactos vão muito além da água acumulada nas ruas. Afetam diretamente a vida de milhares de pessoas, comprometendo moradias, comércios e serviços essenciais. Estimativas da Defesa Civil do Recife apontam que os prejuízos anuais ultrapassam R\$ 100 milhões, atingindo mais de 100 mil moradores em diferentes bairros \cite{defesacivil2023relatorio}. A cada novo episódio, multiplicam-se as perdas materiais e humanas, evidenciando a fragilidade da infraestrutura urbana e a ausência de mecanismos preventivos eficazes.  

Nesse contexto, a falta de um sistema de monitoramento e alerta em tempo real agrava ainda mais a situação. Tanto a população quanto os gestores públicos acabam reagindo de forma tardia, quando os danos já estão em curso. Surge, portanto, a necessidade de soluções tecnológicas que combinem ciência de dados, inteligência artificial e geoprocessamento para antecipar riscos e apoiar a tomada de decisão.

\section{Problema de Pesquisa}

Diante desse cenário, a questão central que orienta este trabalho é: \textit{como desenvolver um sistema inteligente capaz de prever e monitorar, em tempo real, os alagamentos urbanos no Recife, integrando dados meteorológicos, históricos e geográficos, de modo a emitir alertas preventivos à população e aos órgãos de defesa civil?}

\section{Objetivos}

\subsection{Objetivo Geral}

Desenvolver um sistema inteligente para previsão e monitoramento em tempo real de alagamentos urbanos na cidade do Recife, utilizando técnicas de inteligência artificial e geoprocessamento.

\subsection{Objetivos Específicos}

\begin{itemize}
    \item Projetar e implementar uma arquitetura de software escalável para ingestão e processamento de dados meteorológicos em tempo real;
    \item Desenvolver e calibrar algoritmos de previsão híbrida, combinando modelos físico-probabilísticos e de aprendizado de máquina;
    \item Construir uma interface web interativa com mapas de risco georreferenciados para visualização dinâmica das áreas vulneráveis;
    \item Validar o sistema com séries históricas de precipitação e dados reais fornecidos pela APAC e pelo INMET;
    \item Documentar a metodologia para possibilitar a replicação da solução em outras regiões metropolitanas.
\end{itemize}
