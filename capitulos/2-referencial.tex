\chapter{Referencial Teórico}

\section{Alagamentos Urbanos: Conceitos e Impactos}

Os alagamentos urbanos são fenômenos que ocorrem quando a intensidade da chuva ultrapassa a capacidade de drenagem da infraestrutura disponível, provocando o acúmulo rápido de água em ruas, praças e áreas construídas \cite{carvalho2025sistema}. Diferentemente das inundações em grandes bacias hidrográficas, que se desenvolvem ao longo de dias, os alagamentos em áreas urbanas surgem em poucas horas, tornando-se críticos para a mobilidade, a saúde pública e a segurança da população.

No Recife, fatores naturais e humanos se combinam para agravar o problema. O relevo plano, a baixa altitude média em relação ao nível do mar e a influência das marés dificultam o escoamento natural das águas. Ao mesmo tempo, a impermeabilização crescente do solo e as ocupações irregulares em áreas de várzea reduzem a capacidade de infiltração e armazenamento natural \cite{domingos2025mapeamento}. Estudos recentes mostram que a intensificação de eventos extremos de precipitação no Nordeste, com chuvas acima de 50 mm em um único dia, tende a se tornar mais frequente, aumentando o risco de colapso da drenagem urbana \cite{wilson2016analise}.

As consequências vão muito além do desconforto imediato. Estima-se que os prejuízos anuais no Recife ultrapassem centenas de milhões de reais, afetando diretamente milhares de habitantes. Além das perdas materiais, há impactos indiretos, como a interrupção de serviços públicos, a contaminação da água e o aumento de doenças associadas a ambientes alagados \cite{carvalho2025sistema}.

\section{Sistemas de Monitoramento Hidrometeorológico}

O monitoramento das chuvas evoluiu significativamente nas últimas décadas. Se antes dependia de pluviômetros manuais e medições pontuais, hoje conta com redes integradas em tempo real. Em Pernambuco, a Agência Pernambucana de Águas e Clima (APAC) opera dezenas de estações meteorológicas e pluviométricas, que transmitem dados continuamente por meio de tecnologias digitais \cite{alves_apac}.  

Além disso, plataformas em nuvem permitem integrar informações de diferentes fontes, como estações do INMET, imagens de satélite e sensores instalados em pontos estratégicos da cidade. Esses dados alimentam painéis interativos que mostram tendências de chuva, níveis de reservatórios e mapas de suscetibilidade a alagamentos, servindo de base para alertas automáticos enviados à população.

\section{Tecnologias para Previsão de Desastres Naturais}

A previsão de desastres naturais combina três pilares tecnológicos: sensoriamento remoto, sistemas de informação geográfica (SIG) e inteligência artificial. O SIG é utilizado para processar dados de relevo, uso do solo e hidrografia, gerando mapas de áreas suscetíveis a inundações \cite{carvalho2025sistema}. Já os algoritmos de aprendizado de máquina são treinados para identificar padrões de chuvas intensas e produzir previsões de curto prazo, conhecidas como \textit{nowcasting}.  

Em paralelo, tecnologias emergentes como drones e câmeras fixas permitem monitorar em tempo real pontos críticos da cidade, atualizando modelos hidrodinâmicos que simulam o escoamento da água em áreas densamente urbanizadas.

\section{Estudos de Caso em Cidades Brasileiras}

Diversas cidades brasileiras já implementaram sistemas de alerta para reduzir os impactos dos alagamentos. Em São Paulo, o Centro de Gerenciamento de Emergências Climáticas (CGE) utiliza radares meteorológicos e centenas de pluviômetros automáticos, o que reduziu significativamente o tempo de resposta a chuvas intensas \cite{domingos2025mapeamento}.  

No Rio de Janeiro, a integração de dados de maré e precipitação em uma única plataforma possibilitou mapear pontos de alagamento crônico, subsidiando intervenções na rede de drenagem. Já em Belo Horizonte, sensores instalados em galerias pluviais são conectados a modelos estatísticos de previsão, permitindo que a Defesa Civil seja alertada quando a probabilidade de transbordamento ultrapassa limites críticos em até 24 horas.

\section{Fundamentos de Engenharia de Software Aplicada}

Para que sistemas de alerta funcionem de forma confiável, é necessário aplicar princípios sólidos de engenharia de software. Projetos críticos devem priorizar confiabilidade, escalabilidade e facilidade de manutenção, adotando práticas como arquitetura de microsserviços, APIs bem documentadas e integração contínua.  

Além disso, metodologias ágeis como Scrum e Kanban permitem ciclos de entrega mais curtos e maior capacidade de adaptação a mudanças. O uso de contêineres, como Docker e Kubernetes, garante que os ambientes de desenvolvimento e produção sejam consistentes, reduzindo falhas e aumentando a disponibilidade do sistema.
