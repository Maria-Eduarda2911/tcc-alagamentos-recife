\chapter{Conclusão}

\section{Conclusões}

O desenvolvimento do FloodAI-Recife demonstrou que é possível unir ciência de dados, inteligência artificial e geoprocessamento em uma solução prática para enfrentar um dos maiores desafios urbanos da cidade: os alagamentos. Os resultados confirmaram que:

\begin{enumerate}
    \item A integração de dados em tempo real da API da APAC com registros históricos locais aumenta significativamente a precisão das previsões;
    \item A abordagem baseada em múltiplos fatores (chuva instantânea, acumulado, probabilidade e vulnerabilidade espacial) supera modelos simplistas que consideram apenas a precipitação;
    \item A tecnologia proposta é acessível, de baixo custo e pode ser replicada em outras cidades brasileiras com características semelhantes;
    \item A interface web responsiva e os mapas interativos tornam o sistema intuitivo e facilitam sua adoção tanto por gestores públicos quanto pela população em geral.
\end{enumerate}

\section{Contribuições do Trabalho}

\subsection{Contribuições Teóricas}

Do ponto de vista acadêmico, este trabalho trouxe avanços importantes:
\begin{itemize}
    \item Um modelo de previsão adaptado às especificidades de cidades costeiras e de baixa altitude, como Recife;
    \item Uma metodologia de integração de dados heterogêneos (meteorológicos, históricos e geoespaciais) em um único fluxo de análise;
    \item A proposição de um framework para o desenvolvimento de sistemas de \textit{early warning}, que pode servir de base para pesquisas futuras.
\end{itemize}

\subsection{Contribuições Práticas}

Na dimensão prática, as contribuições são igualmente relevantes:
\begin{itemize}
    \item Um sistema funcional, pronto para uso imediato em cenários reais de monitoramento e alerta;
    \item Documentação detalhada que permite a replicação e adaptação da solução em outros contextos urbanos;
    \item A construção de uma base de dados georreferenciada do Recife, que pode apoiar tanto a Defesa Civil quanto futuras pesquisas acadêmicas.
\end{itemize}

\section{Trabalhos Futuros}

\subsection{Recomendações}

Embora os resultados tenham sido promissores, há espaço para evolução. Recomenda-se como trabalhos futuros:

\begin{enumerate}
    \item \textbf{Expansão geográfica}: aplicar o modelo em outras cidades brasileiras, especialmente em capitais costeiras com problemas semelhantes;
    \item \textbf{Aprimoramento técnico}: incorporar modelos mais avançados de \textit{machine learning} e técnicas de \textit{deep learning} para melhorar a acurácia das previsões;
    \item \textbf{Integração institucional}: fortalecer a conexão com órgãos municipais e estaduais, ampliando a utilização do sistema em políticas públicas de prevenção;
    \item \textbf{Alertas multiplataforma}: desenvolver um aplicativo móvel dedicado, com envio de notificações em tempo real para a população.
\end{enumerate}

\bigskip

Em síntese, o FloodAI-Recife mostrou-se uma solução viável, inovadora e de impacto social direto. Ao oferecer previsões confiáveis e alertas acessíveis, o sistema contribui para reduzir danos materiais, preservar vidas e apoiar a gestão pública em situações de emergência. Trata-se de um passo importante rumo a cidades mais resilientes e preparadas para os desafios das mudanças climáticas.
