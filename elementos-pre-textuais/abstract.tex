\begin{abstract}
    This paper presents the development of an intelligent system for real-time prediction and monitoring of urban floods, with specific application in the city of Recife, Brazil. The research addresses the problem of urban flooding that significantly affects quality of life, causing economic and social damages. The proposed system integrates real-time meteorological data from the Pernambuco Water and Climate Agency (APAC) with historical records from Civil Defense, using spatial and temporal analysis algorithms to generate preventive alerts. The system architecture was developed in Python with FastAPI framework, responsive web interface with Leaflet.js, and implements a prediction model based on multiple risk factors. The results demonstrate the system's effectiveness in early identification of critical areas, offering a valuable tool for urban risk management and civil protection.
    
    \vspace{\onelineskip}
    
    \noindent
    \textbf{Keywords}: Urban Flooding. Real-Time Prediction. Intelligent Systems. Recife. Risk Management.
\end{abstract}