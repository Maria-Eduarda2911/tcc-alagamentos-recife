\begin{resumo}
    Este trabalho apresenta o desenvolvimento de um sistema inteligente de previsão e monitoramento de alagamentos urbanos em tempo real, com aplicação específica na cidade do Recife. A pesquisa aborda a problemática dos alagamentos que afetam significativamente a qualidade de vida urbana, causando prejuízos econômicos e sociais. O sistema proposto integra dados meteorológicos em tempo real da Agência Pernambucana de Águas e Clima (APAC) com histórico de ocorrências da Defesa Civil, utilizando algoritmos de análise espacial e temporal para gerar alertas preventivos. A arquitetura do sistema foi desenvolvida em Python com framework FastAPI, interface web responsiva com Leaflet.js, e implementa um modelo de previsão baseado em múltiplos fatores de risco. Os resultados demonstram a eficácia do sistema na identificação precoce de áreas críticas, oferecendo uma ferramenta valiosa para gestão de riscos urbanos e proteção civil.
    
    \vspace{\onelineskip}
    
    \noindent
    \textbf{Palavras-chave}: Alagamentos Urbanos. Previsão em Tempo Real. Sistemas Inteligentes. Recife. Gestão de Riscos.
\end{resumo}