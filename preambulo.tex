% preambulo.tex - VERSÃO CORRIGIDA
% =============================================================================
% PACOTES ESSENCIAIS
% =============================================================================
\usepackage[utf8]{inputenc}   % Codificação do arquivo
\usepackage[T1]{fontenc}      % Codificação da fonte
\usepackage[brazil]{babel}    % Idioma português do Brasil
\usepackage{indentfirst}      % Indentar primeiro parágrafo
\usepackage{graphicx}         % Figuras
\usepackage{float}            % Controle de posição de floats
\usepackage{booktabs}         % Tabelas profissionais
\usepackage{amsmath, amsfonts, amssymb} % Matemática avançada
\usepackage{algorithm}
\usepackage{algpseudocode}
\usepackage{listings}         % Código fonte
\usepackage{xcolor}           % Cores
\usepackage{hyperref}         % Links clicáveis
\usepackage{multirow}         % Tabelas com múltiplas linhas
\usepackage{subfig}           % Subfiguras
\usepackage{siunitx}          % Unidades SI

% =============================================================================
% CONFIGURAÇÕES DE LISTINGS (CÓDIGO)
% =============================================================================
\lstset{
    language=Python,
    basicstyle=\ttfamily\small,
    keywordstyle=\color{blue},
    commentstyle=\color{green!50!black},
    stringstyle=\color{red},
    numbers=left,
    numberstyle=\tiny\color{gray},
    stepnumber=1,
    numbersep=5pt,
    backgroundcolor=\color{white},
    showspaces=false,
    showstringspaces=false,
    showtabs=false,
    frame=single,
    rulecolor=\color{black},
    tabsize=4,
    captionpos=b,
    breaklines=true,
    breakatwhitespace=false,
    escapeinside={\%*}{*)},
    % Permite acentos em português dentro de blocos de código
    literate={á}{{\'a}}1 {ã}{{\~a}}1 {â}{{\^a}}1 {à}{{\`a}}1
             {Á}{{\'A}}1 {Ã}{{\~A}}1 {Â}{{\^A}}1 {À}{{\`A}}1
             {é}{{\'e}}1 {ê}{{\^e}}1 {É}{{\'E}}1 {Ê}{{\^E}}1
             {í}{{\'i}}1 {Í}{{\'I}}1
             {ó}{{\'o}}1 {õ}{{\~o}}1 {ô}{{\^o}}1 {Ó}{{\'O}}1 {Õ}{{\~O}}1 {Ô}{{\^O}}1
             {ú}{{\'u}}1 {Ú}{{\'U}}1
             {ç}{{\c{c}}}1 {Ç}{{\c{C}}}1
}

% =============================================================================
% CONFIGURAÇÕES DE HIPERLINKS
% =============================================================================
\hypersetup{
    colorlinks=true,
    linkcolor=black,
    citecolor=black,
    filecolor=black,
    urlcolor=blue,
    pdftitle={Sistema Inteligente de Previsão de Alagamentos Urbanos},
    pdfauthor={Seu Nome},
    pdfsubject={Trabalho de Conclusão de Curso},
    pdfkeywords={alagamentos, previsão, sistema inteligente, Recife}
}

% =============================================================================
% COMANDOS PERSONALIZADOS
% =============================================================================
\newcommand{\pythoncode}[1]{\lstinline[language=Python]{#1}}
\newcommand{\apiendpoint}[1]{\texttt{#1}}
\newcommand{\risklevel}[1]{\textbf{#1}}

% =============================================================================
% CONFIGURAÇÕES DE FORMATAÇÃO
% =============================================================================
\setlength{\parindent}{1.5cm}
\setlength{\parskip}{0.3cm}
